\documentclass{article}
\usepackage[linesnumbered, lined, boxed]{algorithm2e} 
\usepackage[]{geometry} 

\begin{document}
\title{Notes on N-Queens Solutions}
\author{Robert Dougherty-Bliss}
\date{\today}
\maketitle

The $n$-queens problem is this: Given an $n \times n$ chessboard, can $n$
queens be placed on it such that no queen can attack another? The answer,
proved by E.~Pauls in 1874, is yes, for $n = 1$ and $n > 3$
\cite{jordanbell07}. We list some notes on the running times of various
solutions.

\section*{Running Times}
\label{sec:running_times}

Here we will list various solutions to the $n$-queens problem and compare their
running times. Included with each are pseudocode algorithms.

\begin{table}[h]
 \caption{Running Time of $n$-Queens Solutions}
 \centering
 \begin{tabular}{lc}
     \hline
     Combinatoric Brute Force & $O(n(n^2!))$ \\
     Row-based Brute Force & $O(n^{n+1})$ \\
     Backtracking & $O(n!)$ \\
     Explicit Solution & $O(n)$ \\
     \hline
 \end{tabular}
\end{table}

\subsection*{Guarded Check}
\label{sub:guarded_check}

Some of the solutions check if a given answer is a solution. Here is an
$O(n^2)$ algorithm to check if this is true.

First, some definitions. The $n \times n$ chessboard is represented as $[0, n-1]
\times [0, n-1]$, indexed by row-column points $(r, c)$. The row $r = 0$ is the
top of the board, and the column $c = 0$ is the leftmost column.

The $k$th sum diagonal on a chess board is the set of points $(r, c)$ on the
board such that $r + c = k$. The $k$th difference diagonal is the set of points
$(r, c)$ on the board such that $r - c = k$. Sum diagonals run from bottom-left
to top-right, and difference diagonals run from top-left to bottom-right.

Queens on a chessboard guard their rows, columns, and diagonals. That motivates
the definition of a \textit{guarded point}. Given a point $P(r, c)$ and a set
of queens $Q$, the point $P$ is guarded by $Q$ iff there exists some $R(a, b)
\in Q$ such that $R$ and $P$ share a column, row, or diagonal. That is, iff $a
+ b = r + c$, $a - b = r - c$, $a = r$, or $b = c$. In the worst-case for a set
of $n$ queens, the running time of checking if a point is guarded is $O(n)$. To
check if an answer is a solution, simply perform this check against all $n$
points, which is $O(n)O(n) = O(n^2)$.

\begin{algorithm}
    \caption{Guarded Check}
    \KwIn{$P(r, c)$, a point, and $Q$, a set of queens}
    \KwOut{True if $P$ is guarded by $Q$, False if not}

    \ForEach{$(a, b)$ in $Q$}{
        \If{$a + b = r + c$ or $a - b = r - c$}{
            \Return{True}\;
        }
        \If{$a = r$ or $b = c$}{
            \Return{True}\;
        }
    }

    \Return{False}\;
\end{algorithm}

\subsection*{Combinatoric Brute Force}
\label{sub:combinatoric_brute_force}

A first attempt to solve the $n$-queens problem might be to try every possible
combination of $n$ distinct points from an $n \times n$ chessboard. This gives
$${n^2 \choose n} = \frac{n^2!}{n!(n^2 - n)!}$$ possible combinations. For all
$n \geq 1$, we have that $n^2 - n \geq 0$, so $n!(n^2 - n)! \geq 1$. So for all
$n \geq 1$, $$\frac{n^2!}{n!(n^2 - n)!} \leq n^2!$$ That is, to enumerate every
possible combination is $O(n^2!)$. But we also need to check if each combination
is a solution, which is $O(n)$. So the total worst-case running time of this
solution is $O(n^2!)O(n^2) = O(n^2(n^2!))$.

\begin{algorithm}
    \caption{Combinatoric Brute Force}
    \KwIn{$n$, a non-negative integer}
    \KwOut{A set of $(r, c)$ points that solve the $n$-queens problem, or an
              empty set if no solution exists}

    $combinations \leftarrow$ $[0, n-1] \times [0, n-1]$\;
    \ForEach{answer in combinations}{
        solved $\leftarrow$ True\;
        \ForEach{point in answer}{
            \If{point is guarded by rest of answer}{
                solved $\leftarrow$ False\;
                break\;
            }
        }

        \If{solved}{
            \Return{answer}\;
        }
    }
\end{algorithm}

\subsection*{Row-Based Brute Force}
\label{sub:row_based_brute_force}

An improvement on the previous technique is to note that no solution has two
queens placed in the same row. Then, restrict the queens from being placed in
the same row. There are $n$ choices for the first row, $n$ for the second, and
so on until the $n$th row. That gives $$\underbrace{n \times n \times n \times
\cdots \times n}_{n \ \rm times} = n^n$$ combinations to check. To enumerate all
of these is $O(n^n)$, but we also need to check each, which is $O(n)$. Thus the
worst case running time is $O(n^n)O(n^2) = O(n^{n+2})$. The solution algorithm is
identical except for the method of generating combinations.

\subsection*{Backtracking}
\label{sub:backtracking}

Another improvment is to use backtracking. This method starts by placing a queen
at $(0, 0)$, then moving to the next row. Once a non-guarded point is found on
that row, place a queen there, and move to the next row. Do this until there are
$n$ queens placed.

If every point on a row is guarded, then move back one row and delete the queen
there. Then attempt to find another non-guarded point in that row. If there are
none, move back another row, and attempt to delete and move the queen there. If
the queen in the first row is moved past the end of its row, then there is no
solution possible.

In the worst case, this will search roughly $n$ points in the first row, then
$n - 1$ in the next, and so on, so the worst-case running time is $O(n!)$.

\begin{algorithm}[H]
    \caption{Backtracking}
    \KwIn{$n$, a non-negative integer}
    \KwOut{A set of $(r, c)$ points that solve the $n$-queens problem, or an
              empty set if no solution exists}

    place queen at $(0, 0)$\;
    (row, col) $\leftarrow$ (0, 0)\;

    \While{number of queens $\neq n$}{
        \If{(row, col) is not guarded}{
            place a queen at (row, col)\;
            (row, col) $\leftarrow$ (row + 1, 0)\;
        } \Else {
            \While{col $= n - 1$}{
                \If{row $= 0$}{
                    \tcc{End of first row.}
                    \Return{no solution}
                }
                \tcc{End of current row; backtrack until we can move the
                previous queen.}
                (row, col) $\leftarrow$ previous queen\;
                delete previous queen\;
            }

            col $\leftarrow$ col + 1\;
        }
    }
\end{algorithm}

\subsection*{Explicit Solution}
\label{sub:explicit_solution}

This solution is due to Pauls, discussed in \cite{jordanbell07}. Pauls breaks
down $n > 3$ by congruence classes modulo 6, then offers a solution for each of
them. This requires one comparison, and then the creation of a number of sets
that are joined together. The number of elements in the sets are linear in $n$,
and so this solution is $O(n)$. Note that for Pauls the board is $[1, n] \times
[1, n]$ instead of $[0, n-1] \times [0, n-1]$. An outline of Pauls proof is
given by \cite{jordanbell07}. Another soluion and proof is given by
\cite{hoffman69}.

\begin{algorithm}[H]
    \caption{Pauls' Explicit Solution}
    \KwIn{$n$, a non-negative integer}
    \KwOut{A set of $(r, c)$ points that solve the $n$-queens problem, or an
              empty set if no solution exists}

    \If{$n = 1$}{
        \Return{$\{(1, 1)\}$}
    }

    \If{$n = 2, 3$}{
        \Return{$\emptyset$}\;
    }

    \Switch{$n \bmod 6$}{
        \Case{0, 4}{
            $A1 = \{(2k, k) : 1 \leq k \leq n/2\}$\;
            $A2 = \{(2k - 1, n/2 + k) : 1 \leq k \leq n/2\}$\;
            \Return{$A1 \cup A2$}\;
        }
        \Case{1, 5}{
            $B1 = \{(n, 1)\}$\;
            $B2 = \{(2k, k + 1) : 1 \leq k \leq (n-1)/2\}$\;
            $B3 = \{(2k - 1, (n+1)/2 + k) : 1 \leq k \leq (n-1)/2\}$\;
            \Return{$B1 \cup B2 \cup B3$}\;
        }
        \Case{2}{
            $C1 = \{(4, 1)\}$\;
            $C2 = \{(n, n/2 - 1)\}$\;
            $C3 = \{(2, n/2)\}$\;
            $C4 = \{(n-1, n/2 + 1)\}$\;
            $C5 = \{(1, n/2 + 2)\}$\;
            $C6 = \{(n - 3, n)\}$\;
            $C7 = \{(n - 2k, k + 1) : 1 \leq k \leq n/2 - 3\}$\;
            $C8 = \{(n - 2k - 3, n/2 + k + 2) : 1 \leq k \leq n/2 - 3\}$\;
            \Return{$C1 \cup C2 \cup C3 \cup C4 \cup C5 \cup C6 \cup C7 \cup C8$}\;
        }
        \Case{3}{
            \Return{$\{(n, n)\} \cup  ($solution for $n - 1)$}\;
        }
    }
\end{algorithm}

\pagebreak

\begin{thebibliography}{9}

\bibitem{jordanbell07}
    Bell, Jordan, and Stevens, Brett,
    2007:
    A survey of known results and research areas for n-queens,
    \emph{Discrete Math},
    \textbf{309},
    1--31.

\bibitem{hoffman69}
    E.J. Hoffman, and J.C. Loessi, and R.C. Moore,
    1969:
    Constructions for the Solution of the m Queens Problem,
    \emph{Math. Mag.},
    \textbf{42},
    66--72

\end{thebibliography}

\end{document}
